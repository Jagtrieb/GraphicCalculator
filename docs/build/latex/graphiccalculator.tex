%% Generated by Sphinx.
\def\sphinxdocclass{report}
\documentclass[letterpaper,10pt,russian]{sphinxmanual}
\ifdefined\pdfpxdimen
   \let\sphinxpxdimen\pdfpxdimen\else\newdimen\sphinxpxdimen
\fi \sphinxpxdimen=.75bp\relax
\ifdefined\pdfimageresolution
    \pdfimageresolution= \numexpr \dimexpr1in\relax/\sphinxpxdimen\relax
\fi
%% let collapsible pdf bookmarks panel have high depth per default
\PassOptionsToPackage{bookmarksdepth=5}{hyperref}

\PassOptionsToPackage{booktabs}{sphinx}
\PassOptionsToPackage{colorrows}{sphinx}

\PassOptionsToPackage{warn}{textcomp}
\usepackage[utf8]{inputenc}
\ifdefined\DeclareUnicodeCharacter
% support both utf8 and utf8x syntaxes
  \ifdefined\DeclareUnicodeCharacterAsOptional
    \def\sphinxDUC#1{\DeclareUnicodeCharacter{"#1}}
  \else
    \let\sphinxDUC\DeclareUnicodeCharacter
  \fi
  \sphinxDUC{00A0}{\nobreakspace}
  \sphinxDUC{2500}{\sphinxunichar{2500}}
  \sphinxDUC{2502}{\sphinxunichar{2502}}
  \sphinxDUC{2514}{\sphinxunichar{2514}}
  \sphinxDUC{251C}{\sphinxunichar{251C}}
  \sphinxDUC{2572}{\textbackslash}
\fi
\usepackage{cmap}
\usepackage[T1]{fontenc}
\usepackage{amsmath,amssymb,amstext}
\usepackage{babel}





\usepackage[Sonny]{fncychap}
\ChNameVar{\Large\normalfont\sffamily}
\ChTitleVar{\Large\normalfont\sffamily}
\usepackage{sphinx}

\fvset{fontsize=auto}
\usepackage{geometry}


% Include hyperref last.
\usepackage{hyperref}
% Fix anchor placement for figures with captions.
\usepackage{hypcap}% it must be loaded after hyperref.
% Set up styles of URL: it should be placed after hyperref.
\urlstyle{same}

\addto\captionsrussian{\renewcommand{\contentsname}{Документация для кода приложения GraphicCalculator}}

\usepackage{sphinxmessages}
\setcounter{tocdepth}{1}



\title{Graphic Calculator}
\date{нояб. 18, 2024}
\release{0.6}
\author{Jagtrieb}
\newcommand{\sphinxlogo}{\vbox{}}
\renewcommand{\releasename}{Выпуск}
\makeindex
\begin{document}

\ifdefined\shorthandoff
  \ifnum\catcode`\=\string=\active\shorthandoff{=}\fi
  \ifnum\catcode`\"=\active\shorthandoff{"}\fi
\fi

\pagestyle{empty}
\sphinxmaketitle
\pagestyle{plain}
\sphinxtableofcontents
\pagestyle{normal}
\phantomsection\label{\detokenize{index::doc}}


\sphinxstepscope


\chapter{src}
\label{\detokenize{src:src}}\label{\detokenize{src::doc}}
\sphinxAtStartPar
Директория, в которой хранятся файлы с основным кодом приложения


\section{main.py}
\label{\detokenize{src:module-main}}\label{\detokenize{src:main-py}}\index{module@\spxentry{module}!main@\spxentry{main}}\index{main@\spxentry{main}!module@\spxentry{module}}\index{GraphicCalculator (класс в main)@\spxentry{GraphicCalculator}\spxextra{класс в main}}

\begin{fulllineitems}
\phantomsection\label{\detokenize{src:main.GraphicCalculator}}
\pysigstartsignatures
\pysigline
{\sphinxbfcode{\sphinxupquote{class\DUrole{w}{ }}}\sphinxcode{\sphinxupquote{main.}}\sphinxbfcode{\sphinxupquote{GraphicCalculator}}}
\pysigstopsignatures
\sphinxAtStartPar
Основной класс графического калькулятора
\begin{quote}\begin{description}
\sphinxlineitem{Параметры}\begin{itemize}
\item {} 
\sphinxAtStartPar
\sphinxstyleliteralstrong{\sphinxupquote{scene}} \textendash{} поле, на котором будут отрисовываться графики

\item {} 
\sphinxAtStartPar
\sphinxstyleliteralstrong{\sphinxupquote{CurrentFunction}} \textendash{} объект класса {\hyperref[\detokenize{src:main.MathFunction}]{\sphinxcrossref{\sphinxcode{\sphinxupquote{MathFunction}}}}}, который обрабатывается в момент работы приложения

\item {} 
\sphinxAtStartPar
\sphinxstyleliteralstrong{\sphinxupquote{scale}} \textendash{} масштаб координатной плоскости

\item {} 
\sphinxAtStartPar
\sphinxstyleliteralstrong{\sphinxupquote{PPS}} (\sphinxstyleliteralemphasis{\sphinxupquote{pixels per step}}) \textendash{} отношение количества пикселей в self.scene к единице координатной плоскости

\item {} 
\sphinxAtStartPar
\sphinxstyleliteralstrong{\sphinxupquote{correctiveX}} \textendash{} смещение координатной плоскости по оси X

\item {} 
\sphinxAtStartPar
\sphinxstyleliteralstrong{\sphinxupquote{correctiveY}} \textendash{} смещение координатной плоскости по оси Y

\end{itemize}

\end{description}\end{quote}
\index{coords\_to\_pix() (метод main.GraphicCalculator)@\spxentry{coords\_to\_pix()}\spxextra{метод main.GraphicCalculator}}

\begin{fulllineitems}
\phantomsection\label{\detokenize{src:main.GraphicCalculator.coords_to_pix}}
\pysigstartsignatures
\pysiglinewithargsret
{\sphinxbfcode{\sphinxupquote{coords\_to\_pix}}}
{\sphinxparam{\DUrole{n}{raw}}}
{}
\pysigstopsignatures
\sphinxAtStartPar
Перевод значения на координатной плоскости в значение пикселя в \sphinxstyleemphasis{self.scene}
\begin{quote}\begin{description}
\sphinxlineitem{Параметры}
\sphinxAtStartPar
\sphinxstyleliteralstrong{\sphinxupquote{raw}} \textendash{} Значение координаты на координатной плоскости

\sphinxlineitem{Результат}
\sphinxAtStartPar
Координата пикселя в \sphinxstyleemphasis{self.scene}

\sphinxlineitem{Тип результата}
\sphinxAtStartPar
float

\end{description}\end{quote}

\end{fulllineitems}

\index{draw\_function() (метод main.GraphicCalculator)@\spxentry{draw\_function()}\spxextra{метод main.GraphicCalculator}}

\begin{fulllineitems}
\phantomsection\label{\detokenize{src:main.GraphicCalculator.draw_function}}
\pysigstartsignatures
\pysiglinewithargsret
{\sphinxbfcode{\sphinxupquote{draw\_function}}}
{}
{}
\pysigstopsignatures
\sphinxAtStartPar
Функция, которая отрисовывает математическую функцию

\end{fulllineitems}

\index{draw\_grid() (метод main.GraphicCalculator)@\spxentry{draw\_grid()}\spxextra{метод main.GraphicCalculator}}

\begin{fulllineitems}
\phantomsection\label{\detokenize{src:main.GraphicCalculator.draw_grid}}
\pysigstartsignatures
\pysiglinewithargsret
{\sphinxbfcode{\sphinxupquote{draw\_grid}}}
{}
{}
\pysigstopsignatures
\sphinxAtStartPar
Отрисовывает сетку координатной плоскости на виджете \sphinxstyleemphasis{self.scene}

\end{fulllineitems}

\index{drawing\_procedure() (метод main.GraphicCalculator)@\spxentry{drawing\_procedure()}\spxextra{метод main.GraphicCalculator}}

\begin{fulllineitems}
\phantomsection\label{\detokenize{src:main.GraphicCalculator.drawing_procedure}}
\pysigstartsignatures
\pysiglinewithargsret
{\sphinxbfcode{\sphinxupquote{drawing\_procedure}}}
{}
{}
\pysigstopsignatures
\sphinxAtStartPar
Функция, запускающая процессы отрисовки сетки и графиков

\end{fulllineitems}

\index{fix\_multiply() (метод main.GraphicCalculator)@\spxentry{fix\_multiply()}\spxextra{метод main.GraphicCalculator}}

\begin{fulllineitems}
\phantomsection\label{\detokenize{src:main.GraphicCalculator.fix_multiply}}
\pysigstartsignatures
\pysiglinewithargsret
{\sphinxbfcode{\sphinxupquote{fix\_multiply}}}
{\sphinxparam{\DUrole{n}{raw\_mult}}}
{}
\pysigstopsignatures
\sphinxAtStartPar
Функция для замены умножения в виде 2x или x(…) на 2 * x и x * (…), для правильного прочтения введённой математической функции
\begin{quote}\begin{description}
\sphinxlineitem{Параметры}
\sphinxAtStartPar
\sphinxstyleliteralstrong{\sphinxupquote{raw\_mult}} \textendash{} Введённая (\sphinxstyleemphasis{сырая}) математическая функция

\sphinxlineitem{Результат}
\sphinxAtStartPar
Введённая математическая функция, которая была исправлена для корректного прочтения и обработки

\end{description}\end{quote}

\end{fulllineitems}

\index{pix\_to\_coord() (метод main.GraphicCalculator)@\spxentry{pix\_to\_coord()}\spxextra{метод main.GraphicCalculator}}

\begin{fulllineitems}
\phantomsection\label{\detokenize{src:main.GraphicCalculator.pix_to_coord}}
\pysigstartsignatures
\pysiglinewithargsret
{\sphinxbfcode{\sphinxupquote{pix\_to\_coord}}}
{\sphinxparam{\DUrole{n}{raw}}}
{}
\pysigstopsignatures
\sphinxAtStartPar
Перевод координаты пикселя в self.scene в значение координаты в координатной плоскости
\begin{quote}\begin{description}
\sphinxlineitem{Параметры}
\sphinxAtStartPar
\sphinxstyleliteralstrong{\sphinxupquote{raw}} \textendash{} Значение координаты пикселя в \sphinxstyleemphasis{self.scene}

\sphinxlineitem{Результат}
\sphinxAtStartPar
Значение координаты в координатной плоскости

\sphinxlineitem{Тип результата}
\sphinxAtStartPar
float

\end{description}\end{quote}

\end{fulllineitems}

\index{revise\_function() (метод main.GraphicCalculator)@\spxentry{revise\_function()}\spxextra{метод main.GraphicCalculator}}

\begin{fulllineitems}
\phantomsection\label{\detokenize{src:main.GraphicCalculator.revise_function}}
\pysigstartsignatures
\pysiglinewithargsret
{\sphinxbfcode{\sphinxupquote{revise\_function}}}
{}
{}
\pysigstopsignatures
\sphinxAtStartPar
Функция для проверки введённой математической функции и её отрисовка при удовлетворительном результате

\end{fulllineitems}

\index{select\_func\_color() (метод main.GraphicCalculator)@\spxentry{select\_func\_color()}\spxextra{метод main.GraphicCalculator}}

\begin{fulllineitems}
\phantomsection\label{\detokenize{src:main.GraphicCalculator.select_func_color}}
\pysigstartsignatures
\pysiglinewithargsret
{\sphinxbfcode{\sphinxupquote{select\_func\_color}}}
{}
{}
\pysigstopsignatures
\sphinxAtStartPar
Функия для выбора цвета графика \sphinxstyleemphasis{self.CurrecntFunction}

\end{fulllineitems}


\end{fulllineitems}

\index{MathFunction (класс в main)@\spxentry{MathFunction}\spxextra{класс в main}}

\begin{fulllineitems}
\phantomsection\label{\detokenize{src:main.MathFunction}}
\pysigstartsignatures
\pysiglinewithargsret
{\sphinxbfcode{\sphinxupquote{class\DUrole{w}{ }}}\sphinxcode{\sphinxupquote{main.}}\sphinxbfcode{\sphinxupquote{MathFunction}}}
{\sphinxparam{\DUrole{n}{function}}\sphinxparamcomma \sphinxparam{\DUrole{n}{str\_function}}\sphinxparamcomma \sphinxparam{\DUrole{n}{color=\textless{}PyQt6.QtGui.QColor object\textgreater{}}}}
{}
\pysigstopsignatures
\sphinxAtStartPar
Класс математической функции, хранящий в себе Математическую функцию в двух формах (для расчёта и для демонстрации), а также цвет, которым будет отрисован график функции

\end{fulllineitems}



\section{ui\_file.py}
\label{\detokenize{src:module-ui_file}}\label{\detokenize{src:ui-file-py}}\index{module@\spxentry{module}!ui\_file@\spxentry{ui\_file}}\index{ui\_file@\spxentry{ui\_file}!module@\spxentry{module}}

\renewcommand{\indexname}{Содержание модулей Python}
\begin{sphinxtheindex}
\let\bigletter\sphinxstyleindexlettergroup
\bigletter{m}
\item\relax\sphinxstyleindexentry{main}\sphinxstyleindexpageref{src:\detokenize{module-main}}
\indexspace
\bigletter{u}
\item\relax\sphinxstyleindexentry{ui\_file}\sphinxstyleindexpageref{src:\detokenize{module-ui_file}}
\end{sphinxtheindex}

\renewcommand{\indexname}{Алфавитный указатель}
\printindex
\end{document}